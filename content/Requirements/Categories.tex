%!TEX root = ../../documentation.tex

\section{Requirement categories}\label{cha:requirement_categories}

During the requirements research it became apparent that requirements can be divided into categories.
Grouping requirements by categories further structures the analysis.
This also allows to order them based on there importance.
Thus, the following list shows an interpretation of requirement categories and their importance in descending order.
Each category is abbreviated, except \textit{Performance}, as seen in the following list.
They are also used in the Table \ref{tbl:overview_requirements}.

\pagebreak

\begin{enumerate}
    \item Browser routing and integration approach \textit{(Integration)}
    \item Performance
    \item Shared state and micro frontend inter-communication techniques \textit{(State)}
          \begin{itemize}
              \item Internationalization
              \item Authentication
          \end{itemize}
    \item Styling handling \textit{(Style)}
    \item Developer experience \textit{(Developer)}
          \begin{itemize}
              \item Tools
              \item Debugging
              \item Testing
          \end{itemize}
\end{enumerate}

To challenge the interpretation, each expert was asked to validate it.
This was the third question of the expert interviews (see Appendix \ref{cha:appendix_expertinterview_questions}).
All experts agreed with the selected categories, but the order of importance was slightly different.
An overview of all opinions about the importance is shown in Table \ref{tbl:overview_requirements_categories}.
The number on the left indicates the score.

\begin{table}[h]
    \setlength{\tabcolsep}{4.265pt} % Default value: 6pt
    \begin{tabularx}{\linewidth}{|l|l|l|l|l|l|l|}
        \hline
        \textbf{S}                              &
        \textbf{LM} \cite{Vogel.2020.Mezzalira} &
        \textbf{PR} \cite{Vogel.2020.Rehm}      &
        \textbf{PH} \cite{Vogel.2020.Huber}     &
        \textbf{MS} \cite{Vogel.2020.Steyer}    &
        \textbf{BO} \cite{Vogel.2020.Olleck}    &
        \textbf{IJ} \cite{Vogel.2020.Jovanovic}
        \\ \hline
        1
                                                &
        Integration
                                                &
        Integration
                                                &
        Integration
                                                &
        Integration
                                                &
        Integration
                                                &
        Performance
        \\ \hline
        2
                                                &
        State
                                                &
        Performance
                                                &
        State
                                                &
        Performance
                                                &
        State
                                                &
        Integration
        \\ \hline
        3
                                                &
        Developer
                                                &
        State
                                                &
        Performance
                                                &
        State
                                                &
        Performance
                                                &
        State
        \\ \hline
        4
                                                &
        Performance
                                                &
        Style
                                                &
        Style
                                                &
        Style
                                                &
        Style
                                                &
        Style
        \\ \hline
        5
                                                &
        Style
                                                &
        Developer
                                                &
        Developer
                                                &
        Developer
                                                &
        Developer
                                                &
        Developer
        \\ \hline
    \end{tabularx}
    \caption{Overview of the different opinions about requirement categories}
    \label{tbl:overview_requirements_categories}
\end{table}

% score calculation explanation
In order to determine the order of importance based on the expert's opinion, a score is calculated.
Each position is weighted with a number equal to the order, which is the first column \textit{S}.
The most important is equal to 1, the second is equal to 2 and so on.
The final score of each category is determined by averaging all scores.
The results are shown in Table \ref{tbl:category_scores}.
In conclusion, \textit{Integration} is the most important category.
Following that is the \textit{Performance} and \textit{State} on second, followed by \textit{Style} and lastly \textit{Developer}.
The result is close to the initially predicted order of importance.

\begin{table}[h]
    \newcolumntype{s}[1]{>{\hsize=#1\hsize}X}
    \begin{tabularx}{\linewidth}{|X|X|X|X|X|}
        \hline
        \textbf{Integration} &
        \textbf{Performance} &
        \textbf{State}       &
        \textbf{Style}       &
        \textbf{Developer}
        \\ \hline
        $ \approx 1,2$
                             &
        $ \approx 2,5$
                             &
        $ \approx 2,5$
                             &
        $ \approx 4,2$
                             &
        $ \approx 4,7$
        \\ \hline
    \end{tabularx}
    \caption{Requirement categories scores based on expert opinions}
    \label{tbl:category_scores}
\end{table}
