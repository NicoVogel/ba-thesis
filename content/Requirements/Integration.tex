%!TEX root = ../../documentation.tex

\section{Integration requirements and approaches}\label{cha:requirement_detail_integration}

This section explains the integration requirements.
The requirements from \textit{\nameref{cha:requirement_detail_integration_loading}} until \textit{\nameref{cha:requirement_detail_integration_sharedlogic}} are described in greater detail, than the requirements from \textit{\nameref{cha:requirement_detail_integration_widget}} to \textit{\nameref{cha:requirement_detail_integration_extensible}}, due to their score.





\subsection{Loading}\label{cha:requirement_detail_integration_loading}

The \textit{\nameref{cha:requirement_detail_integration_loading} (\textbf{Score 14})} requirement is about the shell's ability to load a \ac{MF} when needed.
Before describing the approaches, it is important to consider what data needs to be loaded.
A \ac{MF} consists of at least one \ac{JS} file.
Other files are for instance, \ac{CSS}, \ac{HTML}, \ac{JSON} and image files.
Depending on the \ac{MF} the number of files which need to be loaded may vary.
Therefore, the loading approach must be flexible.
End-users also expect the application to be fast \cite[p.~244p]{Zhou.2008}.
Thus, the number of files needed for running the \ac{MF} should be small.
Finally, the shell application needs to know which files need to be loaded.
This is further discussed in the \textit{\nameref{cha:requirement_detail_integration_configuration}} requirement.
There are two approaches on loading a \ac{MF} as well as an optional approaches that can be used on top.

The first approach uses the browser \ac{API} to load the files.
\textciteMezzalira{} proposes that one \ac{HTML} file per \ac{MF} is exposed and it contains all tags needed to load and startup the application.
A little \ac{JS} code is responsible to load the \ac{HTML} file and integrate all tags into the application.
The Browser then automatically starts downloading the needed files.
Another variant of this, which also utilizes the browser \ac{API}, is to add each loading tag individually in the current application.
This was mentioned by \textcite{Dornenburg.2019} and \textcite{Laug.2018}.

The second approach only addresses \ac{JS} files.
It uses \ac{JS} module import for loading the files.
As soon as the file is fully loaded, it is executed.
So, in this approach no tags are added to the current application and the \ac{JS} file is only loaded and executed within \ac{JS} \cite{Vogel.2020.Olleck}.
According to \citeauthorMezzalira{} it is quicker to let the browser load the files, than executing \ac{JS} upfront to load it.

Finally, on top of the named approaches a Service Worker can be used for caching and preloading, which was mentioned by \textciteRehm{} and \textciteSteyer{}.
Service Workers run in a separate thread from the \ac{UI} and can cache images, scripts, styles or even pages \cite[p.~24f.]{Sheppard.2017}.

This requirement has the highest score for the Integration category.
It is considered crucial from \textciteHuber{} and difficult from \textciteMezzalira{}, \textciteSteyer{}, \textciteOlleck{} and \textciteJovanovic{}.





\subsection{Lifecycle management}\label{cha:requirement_detail_integration_lifecycle}

The requirement \textit{\nameref{cha:requirement_detail_integration_lifecycle} (\textbf{Score 10})} is an extension to the \textit{\nameref{cha:requirement_detail_integration_integration}} requirement.
In general \textit{\nameref{cha:requirement_detail_integration_integration}} is about which technique is used to include a \ac{MF} into the application, the \textit{\nameref{cha:requirement_detail_integration_lifecycle}} is about the \acp{MF} visibility and loading state.
A \ac{MF} lifecycle can consist of multiple events which represent its current state.
These events could be triggered by the shell application or by the \ac{MF} itself.
\textciteOlleck{} named states he used in a project, which are \textit{bootstrap}, \textit{display}, \textit{hide} and \textit{destroy (or clear)}.
\textit{Bootstrap} represents the integration and startup, \textit{display} represents that a \ac{MF} is shown to the user, \textit{hide} is the opposite of display and \textit{destroy} or \textit{clear} represents that the \ac{MF} is unloaded and its occupied resources are free.
Note, that this is one example and not all applications need such an extensive lifecycle.
Another example from \textciteRehm{} provides only the \textit{bootstrap} event, which is raised when a \ac{MF} is in the integration process.
\citeauthorRehm{} and \textciteSteyer{} state that lifecycle management an optional feature.

For this requirement, no explicit approach is mentioned other than raising events and this is part of the \textit{\nameref{cha:requirement_detail_state_intercommunication}} requirement.
The only explicit description of a functionality is, that in case of a destroy or clear event the \ac{MF} must provide a function which is called to remove all event listener.
\textciteMezzalira{} points out, that in this case an interface which provides such a function is needed.
This requirements was mentioned as difficult from \textciteJovanovic{} and \citeauthorMezzalira{}, and as crucial from \textciteHuber{}.





\subsection{Routing}\label{cha:requirement_detail_integration_routing}

\textit{\nameref{cha:requirement_detail_integration_routing} (\textbf{Score 9})} is about
\ac{URL} changes that are reflected in the \ac{UI}.
Updating the \ac{UI} consists of either changing the current page layout or swapping \acp{MF} accordingly.
The idea of changing the page layout based on the route is further described in the \textit{\nameref{cha:requirement_detail_integration_pagelayout}} requirement.

There is an important first differentiation in terms of how the application functions.
The differentiation depends on the business domain size which each \ac{MF} covers.
But the important part for routing is, that a \ac{MF} either has only one page or multiple pages.
In the first case, only the shell needs routing functionality, while the latter is a combination of both.
No matter which approach is used, all experts mention splitting the \ac{URL} into different sections.
These sections have different responsibilities.
The base route of the \ac{URL} is mentioned in the following two examples.
It is the value between the first and second slash in the \ac{URL}.

\textcite{Grijzen.2019} provides an example where only the shell is responsible for routing.
No implementation details are shared, but his example is combined with the \textit{\nameref{cha:requirement_detail_integration_pagelayout}} requirement.
The base route determines which \ac{MF} is shown, while a query parameter\footnotemark  is added to save the page layout and distribution of other \acp{MF}.
\footnotetext{\url{https://developer.mozilla.org/en-US/docs/Web/API/URLSearchParams} (Visited on 09/06/2020)}

Another example provided by \textciteRehm{} and \textciteSteyer{} addresses the idea of splitting the routing responsibility between the shell and \acp{MF}.
In the example, the shell manages the base route.
Based on this value one \ac{MF} is shown and it is responsible for the rest of the \ac{URL}.
Therefore, it provides its own router that handles page changes which are within itself.
This results in a hierarchical setup, where the shell sits on top and is essentially a meta router between \acp{MF}, which in turn routes on their own.

\textciteSteyer{} points out, that there are some difficulties with this approach and it boils down to the fact, that the browser \ac{API} does not support an event for \enquote{\ac{URL} has changed}.
He goes on to say, that there is a workaround for this problem, which is monkey patching\footnotemark the browser to support such an event.
\footnotetext{\url{https://web.archive.org/web/20080604220320/http://plone.org/documentation/glossary/monkeypatch} (Visited on 09/06/2020)}
The only browser \ac{API} event which is close to the needed event is \textit{HashChangeEvent}\footnotemark{}.
\footnotetext{\url{https://www.w3schools.com/jsref/obj_hashchangeevent.asp} (Visited on 08/12/2020)}
It is fired when the hash value in the \ac{URL} changes.
However, \citeauthorSteyer{} points out that not every customer wants this approach.
\textciteRehm{} tackled this problem by creating navigation events, which are distributed between all \acp{MF} and the shell.
The event is raised when a \ac{MF} or shell changes the \ac{URL}, so that other \acp{MF} or the shell are notified about the change.
More on events is described in the \textit{\nameref{cha:requirement_detail_state_intercommunication}} requirement.
Lastly,  \textciteJovanovic{} considers \textit{\nameref{cha:requirement_detail_integration_routing}} requirement to be difficult and \citeauthorSteyer{} believes it is critical.





\subsection{Configuration}\label{cha:requirement_detail_integration_configuration}

The requirement \textit{\nameref{cha:requirement_detail_integration_configuration} (\textbf{Score 8})} is the only requirement which can be part of two requirement categories.
On one hand, it is part of \textit{\nameref{cha:requirement_detail_integration_integration}}, because it provides information on how to integrate a \ac{MF} and on the other hand, it improves the \textit{\nameref{cha:requirement_detail_developer_experience}} by simplifying changing behavior.
Both factors will be explained in the following paragraphs.

An application configuration should contain everything which changes between deployments.
To simplify this, some configuration is stored in environment variables \cite{Wiggins.2017}.
In order to access the environment variables in the frontend, they must be stored in a file or they are provided by endpoint.
Another aspect to consider is that configuration removes coupling from code, which is mentioned by \textcite{Dornenburg.2019} and \textciteMezzalira{}.
More about coupling is described in the \textit{\nameref{cha:requirement_detail_developer_autonomy}} requirement.
\citeauthorMezzalira{} says that being able to define which \ac{MF} should be loaded enables practices like Canary releases\footnotemark{} or loading different \ac{MF} per country.
\footnotetext{\url{https://martinfowler.com/bliki/CanaryRelease.html} (Visited on 09/06/2020)}
Other practices which are enabled by such a config, are Blue-Green deployments\footnotemark and quick rollbacks to the previous version, which is mentioned by \textciteRehm{}.
\footnotetext{\url{https://martinfowler.com/bliki/BlueGreenDeployment.html} (Visited on 09/06/2020)}

Apart from the release aspects, a configuration enables caching for \ac{JS} and other files.
A few conditions must be met in order to achieve this.
First, the configuration must contain \acp{URL} to the \ac{MF} files which need to be loaded.
Then, these files should include a hash value in their names, so that a cache miss occurs in case of a file change.
Finally, the configuration file itself should not be cached and available at a default location which itself does not change.
This approach was mentioned by \textciteOlleck{}, \textcite{Leitner.2020} and \textcite{Dornenburg.2019}.
The result would be, that the \ac{MF} files are loaded once and as soon as the entry within the configuration changes, a cache miss occurs in the browser and the new file is downloaded.
Another approach mentioned by \citeauthor{Leitner.2020} is passing the configuration to the client is to directly include it into the shell \ac{HTML} file.
One way to achieve this is to use Edge-Side-Includes\footnotemark and to update the \enquote{index.html} from the shell which is passed to the user.
\footnotetext{\url{https://www.w3.org/TR/esi-lang/} (Visited on 09/06/2020)}
Therefore, reducing the total request count.

Apart from the \ac{MF} file \acp{URL}, the configuration file could include information like the backend \ac{URL} that each \ac{MF} should use or what the base route of a \ac{MF} is.
This is mentioned by \textciteRehm{}, \textciteOlleck{} and \textcite{Dornenburg.2019}.
Defining the file locations and backend \ac{URL} via environment variables allows the deployment to multiple environments.
\citeauthorOlleck{} points out, that for the configuration information, which is passed to the \ac{MF}, there should be a clear interface.
This overlaps with the \textit{\nameref{cha:requirement_detail_state_exchange}} requirement.
Besides passing information to the \ac{MF}, \textcite{Grijzen.2019} points out, that meta data like a display name for a \ac{MF} could be included as well.
The display name could be used in the navigation panel as label for the \ac{MF}.
This allows to dynamically build the navigation pane without the need to first load the \ac{MF} or create unnecessary coupling between the navigation panel and all \acp{MF}.

This requirement is considered crucial by \textcite{Vogel.2020.Rehm}.
The actual content of the configuration file depends on the scenario.
For example, the scenario \textit{\nameref{cha:scenarios_offshelf}} could be customized mainly via such a configuration file in combination with Blue-Green deployment or other similar processes are a common practice in \textit{\nameref{cha:scenarios_enterprise}}s.





\subsection{Integration}\label{cha:requirement_detail_integration_integration}

The  \textit{\nameref{cha:requirement_detail_integration_integration} (\textbf{Score 7})} requirement revolves around the possibilities to add a \ac{MF} to an application.
While the \textit{\nameref{cha:requirement_detail_integration_lifecycle}} requirement addresses the \ac{MF} visibility and loading state, the \textit{\nameref{cha:requirement_detail_integration_integration}} requirement addresses the technique used to integrate a \ac{MF}.
Before explaining which approaches are available, it is important to point out, that the integration process is the new breaking point of a \ac{MF} application.
\textciteOlleck{} points out, that some errors are only visible at run-time, while they were obvious in monoliths at compile-time.
He believes that an important question is how to get certainty in the integration process, before a user gets to see the new release.
There is no clear answer to this question, but \textcite{Laug.2018} mentioned a possible integration check, which is discussed in the \textit{\nameref{cha:requirement_detail_developer_testing}} requirement.
\citeauthor{Laug.2018} also points out, that the shell is responsible to handle integration errors.

There are three different ways to integrate \acp{MF}.
The first one is using Web Components, mentioned by \textciteMezzalira{}, \textciteRehm{}, \textciteHuber{} and \textciteOlleck{}.
Using Web Components enables the definition of a technology agnostic interface, which uses a browser \ac{API} to integrate itself into the application.
A drawback is that not all browser support Web Components, but the current modern browsers do\footnotemark.
\footnotetext{\url{https://www.webcomponents.org/} (Visited on 08/29/2020)}
This can be partially fixed by using polyfills\footnotemark, but \citeauthorMezzalira{} points out, that they do not work in all browsers.
\footnotetext{\url{https://developer.mozilla.org/en-US/docs/Glossary/Polyfill} (Visited on 09/06/2020)}
For example, some console or TV browsers are not compatible to use Web Components with these polyfills.
\citeauthorOlleck{} adds that the polyfills in Internet Explorer are difficult to work with.
Another drawback is that if the feature shadow \ac{DOM} is used, a search engine crawler will not pickup content which is in the shadow \ac{DOM} as pointed out by \citeauthorMezzalira{}.
More about shadow \ac{DOM} in the section \ref{cha:theory_practice_webcomponent}.

The next approach was mentioned by \textciteMezzalira{} and it uses \ac{IIFE} to integrate a \ac{MF}.
\acp{IIFE} are used to fully encapsulate \ac{JS}.
This technique works in all browsers and is also used by a feature from \textit{Webpack} called \textit{Module Federation}\footnotemark, which was also mentioned \citeauthorMezzalira{}.
\footnotetext{\url{https://webpack.js.org/concepts/module-federation/} (Visited on 09/06/2020)}
The last approach uses iFrames, which allow integrating technology agnostic \acp{MF}.
\textciteSteyer{} mentions that using iFrames comes with some overhead.
He solved that problem, by implementing an abstraction in the shell to managing iFrames.
He further says that he used this approach before Web Components were as broadly supported as nowadays.
Because the browser support is now there, he will probably use Web Components instead of iFrames for future projects.
Finally, the \textit{\nameref{cha:requirement_detail_integration_integration}} requirement is considered difficult from \textciteOlleck{}.




\subsection{Shared logic}\label{cha:requirement_detail_integration_sharedlogic}

\textit{\nameref{cha:requirement_detail_integration_sharedlogic} (\textbf{Score 6})} is about including logic or \ac{UI} into the shell application.
The shared logic can be accessed by all \acp{MF} and the shared \ac{UI} elements are displayed by the shell instead of the \acp{MF}.

The opinions about what should be shared in the shell vary between the experts.
On the one hand, \textciteRehm{}, \textciteOlleck{}, \textcite{Grijzen.2019} and \textcite{Dornenburg.2019} support the idea of sharing logic and \ac{UI} elements.
An example was brought up by  \textciteSteyer{}, which is sharing the navigation or footer panel, because it is the same for the entire application.
Another example was mentioned by \citeauthorRehm{} and \citeauthor{Dornenburg.2019}, where a toast feature is shared via the shell.
The shell listens for a specific event to show a toast, which can be invoked by any \ac{MF}.
Therefore, it is a combination of shared logic and \ac{UI}.
The third example for shared logic only was provided by \citeauthorOlleck{}, where the shell provides a service to handle backend calls.
The service has an optional feature to freeze the application and show a loading spinner until the call is finished.
This is a cross cutting concern, which was required for some backend calls and only the shell can handle it.
Eventually, \citeauthor{Grijzen.2019} says that the shell should expose a platform \ac{API} which handles cross cutting concerns.

On the other hand, \textcite{Laug.2018} is against sharing anything through the shell.
Instead there should rather be copy-paste ready code for the developers.
He makes one exception, which is authentication.
Authentication is a critical task and should be handled by the shell, but authorization is up to each \ac{MF}.
More about this in the \textit{\nameref{cha:requirement_detail_state_security}} requirement.
\textciteSteyer{} also advises to share as little business logic via the shell as possible.
He mentioned one exception, and this is, if the bundle size is a concern for the application.
In this case, the shell should be responsible for loading shared dependencies once, like a shared core.
Bundle size is further addressed in the \textit{\hyperref[cha:requirement_detail_performance]{Performance}} requirement.

This requirement was only mentioned by \textciteRehm{}, \textciteSteyer{}, \textciteOlleck{}, \textcite{Dornenburg.2019}, \textcite{Laug.2018} and \textcite{Grijzen.2019}
No expert thinks that it is critical or difficult to realize.



\subsection{Other requirements}

In the following requirements with a score less than 5 are outlined.



\paragraph{Widgets} \label{cha:requirement_detail_integration_widget}

The requirement \textit{\nameref{cha:requirement_detail_integration_widget} (\textbf{Score 4})} evolves around the idea, that a \ac{MF} is able to share components with other \acp{MF}.
These components could be anything from a button to a page.
This requirement is needed if the \ac{UI} needs to handle cross cutting concerns, like an e-commerce application.
An example scenario would be a web shop with a product detail and shopping cart page.
Each page is in another subdomain and thus, in different \acp{MF}

Assume the user is visiting a product and wants to add it to the shopping cart.
The \ac{MF} responsible for the product page does not know the \ac{MF} responsible for the shopping cart.
In order to still allow this cross cutting user action, the product page needs to include this functionality from the shopping cart \ac{MF}.
The solution is that the shopping cart \ac{MF} provides an \enquote{add to cart} button as widget, which can be used and integrated by the product page.

An approach which was mentioned by \textciteRehm{} and \textciteHuber{} is to use Web Components to expose widgets.
The widgets are registered by their parent \ac{MF} and can be used by other \acp{MF} as needed.
This introduces coupling between \acp{MF}, but it is rather low, because the other \acp{MF} only need to know the name of a Web Component and not where it comes from or who exposes it.
However, \citeauthorRehm{} points out, that coupling is the cost for cross cutting \acp{UI} and keeping a good \ac{UX}.
It is important to note, that the shell needs to make sure, that a \ac{MF} is loaded, even if only its widget is needed.
This was mentioned by \citeauthorRehm{} and he also thinks, that the \textit{\nameref{cha:requirement_detail_integration_widget}} requirement is difficult to realize.



\paragraph{Page layout}\label{cha:requirement_detail_integration_pagelayout}

The requirement \textit{\nameref{cha:requirement_detail_integration_pagelayout} (\textbf{Score 4})} introduces a concept of pages which contain more than one \ac{MF}.
There are different extents to which this requirement can be realized.
The simplest form is a tab-based separation of \acp{MF}.
Then there is the tile-based arrangement of multiple \acp{MF} and finally the template-based arrangement.

A tab-based separation was mentioned by \textciteOlleck{}.
The tab functionality is handled by the shell.
The next implementation is the tile-based arrangement.
This was proposed by \citeauthorOlleck{} and \textcite{Laug.2018}.
While \citeauthorOlleck{} explained, that the tile functionality was part of the shell, \citeauthor{Laug.2018} used a separate Web Component to handle this feature solely.
\citeauthor{Laug.2018} did not use the shell for this feature, because he believes that the shell should not contain business logic.
More on that in the \textit{\nameref{cha:requirement_detail_integration_sharedlogic}} requirement.

The last approach was mentioned by \textcite{Grijzen.2019} and \textciteMezzalira{}, but \citeauthorMezzalira{} only referenced to \citeauthor{Grijzen.2019}.
\citeauthor{Grijzen.2019} explained an approach which uses templates to determine the page layout.
These templates contain placeholders which can be filled with a \acp{MF}.
The template handling and filling of the placeholders is included into the shell.
As described in the \textit{\nameref{cha:requirement_detail_integration_routing}} requirement, \citeauthor{Grijzen.2019} uses a routing mechanism which changes the current active \ac{MF} based on the base route.
Therefore, the \ac{URL} only reflects the current \ac{MF}.
If the \ac{URL} is stored in a bookmark or shared with a colleague, then the layout is missing.
Hence, the inclusion of a state query parameter is required.
The value is an encoded reflection of the current template and all \acp{MF} that are placed in it.
Consequently, the state query parameter allows to save or share a \ac{URL}.



\paragraph{Indexable via crawler}\label{cha:requirement_detail_integration_crawler}

The requirement \textit{\nameref{cha:requirement_detail_integration_crawler} (\textbf{Score 2})} is relevant for \nameref{cha:scenarios_consumer}, because it probably is a publicly available application.
If an application is indexable via a crawler, then its page content can be analyzed by search engine crawlers.
This analysis is required to make it available in the respective search engine \cite[p.~1]{Qureshi.2010}.

\textciteMezzalira{} pointed out, that current crawler do not support content which is rendered inside the shadow \ac{DOM}.
To validate this claim, it is important to first check if the search engines support \ac{JS} rendered content, because the main approach of this thesis is Client-side integration, which implies \ac{JS} rendered content.
Only \textit{Google} and \textit{Ask} are able to index such content\footnotemark{}.
\footnotetext{\url{https://moz.com/blog/search-engines-ready-for-javascript-crawling} (Visited on 08/17/2020)}
When also considering the usage statistics of search engines, only \textit{Google} remains as a possible candidate, able to index Web Components\footnotemark{}.
\footnotetext{\url{https://www.oberlo.com/statistics/search-engine-market-share} (Visited on 08/17/2020)}

The \textit{Google} crawler does support Web Components, but the content inside a shadow \ac{DOM} must be appended into its slot (see \ref{cha:theory_practice_webcomponent}).
Otherwise the content is not picked up by the crawler\footnotemark{}.
\footnotetext{\url{https://developers.google.com/search/docs/guides/javascript-seo-basics\#web-components} (Visited on 08/17/2020)}
Therefore, populated content from the Web Component itself needs to be appended into the slot, if it needs to be picked up.
This is not always possible, especially when the slot is used as intended.

This requirement was mentioned by \textciteMezzalira{} and \textcite{Dornenburg.2019}.



\paragraph{\ac{MF} interoperable}\label{cha:requirement_detail_integration_interoperable}

In \nameref{cha:scenarios_enterprise}s there can be a requirement, that a \ac{MF} needs to be interoperable (\textbf{Score 2}).
There are two interpretations of this requirement.
Either the \ac{MF} is used in multiple applications or it is used in multiple locations in the same application.

In order to achieve reusability over multiple applications, an \ac{API} definition is needed.
This definition must be the same for both applications, so that a \ac{MF} can simply be integrated in both.
Such an \ac{API} needs to provide everything that the \ac{MF} needs.
The second interpretation can be easily be achieved if the \ac{MF} consists of only one page.
But if the \ac{MF} consists of multiple pages, this would require a flexible router within the \ac{MF}.
Both approaches are described in the \textit{\nameref{cha:requirement_detail_integration_routing}} requirement.
Also, it is common to use a base route to determine which \ac{MF} should be displayed.
Hence, if the \ac{MF} has an internal router, it must support that the base route is passed in as property, in order to support the \textit{\nameref{cha:requirement_detail_integration_interoperable}} requirement.
Both approaches were mentioned by \textciteRehm{}, who works on an \nameref{cha:scenarios_enterprise}.

Another impulse for this requirement was provided by \textcite{Grijzen.2019}.
He works on a \nameref{cha:scenarios_offshelf} and it has a standard technology stack (platform) as base.
This allows to create third party extensions, which can be included into any of the \nameref{cha:scenarios_offshelf}.
More on this approach in \textit{\nameref{cha:requirement_detail_integration_extensible}}.



\paragraph{Abstraction layer}\label{cha:requirement_detail_integration_abstraction}

The requirement \textit{\nameref{cha:requirement_detail_integration_abstraction} (\textbf{Score 1})} was mentioned by \textciteMezzalira{}.
He pointed out, that the application DAZN is used by multiple platforms.
These platforms do not all share the same browser \ac{API}.
In this case, the application shell can be used as abstraction layer for all \ac{API} functionalities which differ between platforms.
To achieve this, there are two approaches:

Either there is one application shell per platform or only a part of the shell is loaded depending on the platform.
An example for the latter would be, that an interface is defined, which handles all differing browser \ac{API} calls.
The interface is then implemented per platform and bundled into a library.
Following that, the library is dynamically loaded depending on the platform.



\paragraph{Third party extensible}\label{cha:requirement_detail_integration_extensible}

In case of a \nameref{cha:scenarios_offshelf}, a requirement can be that it is  \textit{\nameref{cha:requirement_detail_integration_extensible} (\textbf{Score 1})}.
Therefore, anyone can create a \ac{MF} for the application, which can be included if needed.
The requirement \textit{\nameref{cha:requirement_detail_integration_interoperable}} provides more information about reusable \acp{MF}.

\textcite{Grijzen.2019} explains an approach where the third-party extensions are included via an iFrame.
An iFrame is fully encapsulated from the application, like a sandbox.
In order to enable the third-party \ac{MF} to communicate with the host application, an \ac{RPC} like communication library is provided.
\citeauthor{Grijzen.2019} further explains, that the communication library restricts some features for the third-party \ac{MF}.
For example, critical functions are excluded to ensure that extensions do not gain control over the host application.
