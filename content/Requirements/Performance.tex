%!TEX root = ../../documentation.tex

\section{Performance requirement and approaches}\label{cha:requirement_detail_performance}

The \textit{Performance (\textbf{Score 11})} requirement category cannot be divided into sub requirements, but approach categories, which focus around dependencies and integration.



\paragraph{Dependencies}

Approaches for dependencies mainly are based around the idea of reducing the total amount of \ac{JS} loaded.
The following approaches are described from least savings to maximum savings.
The same applies for coupling.
\textciteRehm{} and \textciteSteyer{} mentioned that dependencies or \ac{JS} files should be optimized for loading.
This can be realized with \textit{Webpack minimize}\footnotemark, for example.
\footnotetext{\url{https://webpack.js.org/configuration/optimization/} (Visited on 09/06/2020)}

To reduce  the total size of \ac{JS} to be loaded further, dependencies which are needed from many \acp{MF} can be bundled together and only loaded once.
This dependency bundle is then provided by the shell for all \acp{MF}.
The approach was mentioned by \textcite{Grijzen.2019}, who thinks that this should be utilized as much as possible.
\textciteSteyer{} also believes, that the shell should be responsible for loading the shared dependencies.
Other shared tasks that the shell manages are discussed in the \textit{\nameref{cha:requirement_detail_integration_sharedlogic}} requirement.
In general, this approach is flexible in terms of coupling between \acp{MF}.
The more dependencies that are shared, the higher the coupling between \acp{MF} and the saving of duplicate \ac{JS} code.
While this approach is only theoretical, the next one also explains how to achieve this approach.

To reduce the total \ac{JS} size even more, the frontend framework core can be shared as well.
This was mentioned by \textciteSteyer{}, \textciteRehm{}, \textcite{Grijzen.2019}, \textcite{Dornenburg.2019} and \textcite{Leitner.2020}.
They explained two variances to achieve this with Webpack:

The first one uses the \enquote{external} feature.
In order to extract the framework core from a \ac{MF}, it must be defined as \enquote{external} in the Webpack config.
By doing so, Webpack bundles only the \ac{MF} code and will not include the framework core.
Instead, it expects the core to be accessible via a global variable, which must be defined in the Webpack config\footnotemark{}.
\footnotetext{\url{https://webpack.js.org/configuration/externals/} (Visited on 08/17/2020)}

Another approach to share the core is by using the Module Federation feature of Webpack.
If needed, the core can be lazy loaded and the shell does not need any additional logic to do so\footnotemark{}.
\footnotetext{\url{https://indepth.dev/webpack-5-module-federation-a-game-changer-in-javascript-architecture/} (Visited on 08/17/2020)}
Both approaches were mentioned by \citeauthorSteyer{} and both can be used for either sharing a framework core or dependencies.

Using the shared core approach has the advantage of drastically reducing the total \ac{JS} size, but only when using a heavy framework like Angular, React or Vue.
\textcite{Leitner.2020} points out, that in case of slim frameworks like Stencil or SvelteJS, it does not impact the bundle size as much.
On the other hand, sharing the core introduces a tight coupling between the \acp{MF} and thus, reducing the autonomy of each team.
More about this in the \textit{\nameref{cha:requirement_detail_developer_autonomy}} requirement.

The experts do have different opinions on using the shared core strategy.
These differences are related to the type of application that is built.
For example, \textciteSteyer{} supervised multiple \nameref{cha:scenarios_enterprise}s, where there was no need to keep the bundle size small or even worrying about performance.
\textcite{Laug.2018} also mentions that bundle size is no concern in the \nameref{cha:scenarios_enterprise} he was working on, because the end-users are either frequent users and thus caching reduces the loading time or the devices are connected via \ac{LAN} which typically implies a good internet connection.
Therefore, both did not need to use the shared core approach.

\textcite{Dornenburg.2019} did not mention which application type he was working on, but he suggested not to share multiple versions of the same framework.
By using the described approaches, it is possible to share as many framework cores and versions as needed.
However, it is important to note that sharing multiple cores also reduces the size benefit of sharing cores at all.
The end user still needs to download these cores.
Therefore, \textcite{Grijzen.2019} points out, that for the \nameref{cha:scenarios_offshelf} he is working on, only one framework core is shared with only one version of it.

Each of the approaches, namely minimizing \ac{JS} bundles, sharing dependencies, and sharing the framework core, can be further improved via caching.
\textciteSteyer{} suggests using Service Workers to cache \ac{MF} files.
This is relevant for loading required files, which is further discussed in the \textit{\nameref{cha:requirement_detail_integration_loading}} requirement.
\textciteRehm{} believes, that a caching strategy is needed.
An approach to allow caching files, is mentioned in the \textit{\nameref{cha:requirement_detail_integration_configuration}} requirement.
It focuses around the file names and which files should be cached and which should not.
\citeauthorRehm{} further mentions to use a loading strategy.
This is needed if the \textit{\nameref{cha:requirement_detail_integration_widget}} requirement applies to the application.
In this case, a widget could be used, before the parent \ac{MF} is loaded.
This is further discussed in the requirement itself.
Lastly, it is important to mention that caching is especially useful if the userbase consists of frequent users.

To conclude, dependencies should be minimized and adding a form of caching improves loading performance for frequent users.
Also, using shared dependencies or even a shared framework core approach, introduces coupling between the \acp{MF}.
Depending on the application type, however, using shared dependencies or core is necessary for the \ac{UX}.
While it may be negligible for \nameref{cha:scenarios_enterprise}s \cite{Vogel.2020.Steyer}, it can be critical for \nameref{cha:scenarios_consumer}s or \nameref{cha:scenarios_offshelf}s.

Finally, depending on the application itself, reducing the total \ac{JS} size is not always as important.
When considering an application like Instagram, their focus is around images and videos, which have a greater impact in size than any \ac{JS} could have.
This is also underlined by the \ac{HTTP} Archive, which shows that websites for desktops contain around 3.8 times more bytes for images than \ac{JS}\footnotemark{}.
\footnotetext{\url{https://legacy.httparchive.org/interesting.php?a=All\&l=Aug\%201\%202020\#bytesperpage} (Visited on 08/29/2020)}



\paragraph{Integration}

While the dependency approaches focus on shortening the loading time, the integration approaches address the performance of the application itself.
One approach mentioned by \textciteSteyer{} targets the idea of re-fetching information.
In his opinion, each \ac{MF} should load its needed information by itself and not using the shell as a cache.
If the shell is used to cache information, this may result in a tight coupling between the shell and \acp{MF}.
However, he adds that some context information should be provided by the shell.
For example, who is the current user or which business process is in use.
The named examples from \citeauthorSteyer{} refer to context information which are needed by multiple \acp{MF} to correctly display content.

Another more general topic \textciteSteyer{} mentions is perceived performance of the application.
He states, that he was working on applications with around 15 active \acp{MF} and no performance problems were observed.
The application was not using a shared core approach and it was an \nameref{cha:scenarios_enterprise}.
Moreover, he was involved with testing iFrames performance for another \nameref{cha:scenarios_enterprise}.
He included 200 iFrames in an application and each was displaying another dummy \ac{MF}.
No performance issues were experienced during the test.

On the contrary, \textciteJovanovic{} points out that one of his main issue with \ac{MF} applications is performance.
This is also shown in his importance ordering of the requirement categories.
He rated performance as the most important requirement, which is shown in Table \ref{tbl:overview_requirements_categories}.
\citeauthorJovanovic{} worked on a \nameref{cha:scenarios_consumer}.

\textciteJovanovic{} and \textcite{Leitner.2020} both mentioned that in case of performance issues, one possible solution can be to reduce the number of heavy frameworks per page.
A heavy framework would be Angular, React or Vue in their opinion.
\citeauthor{Leitner.2020} outlines that weak devices will have a hard time running multiple heavy frameworks.
Lastly, \citeauthorJovanovic{} believes that Web Workers\footnotemark could improve the overall performance.
\footnotetext{\url{https://developer.mozilla.org/en-US/docs/Web/API/Web_Workers_API/Using_web_workers} (Visited on 09/06/2020)}
He also pointed out, that this will not fix all problems, because weak devices will still have problems even with a multi-core approach.



\paragraph{Conclusion}

In general performance is a tradeoff between autonomy and \ac{UX}.
This is especially shown in case of the loading time, which can be optimized on the expense of autonomy.
Also, depending on the application scenario the importance of choosing a performant approach varies.
This is further outlined by the following two experts opinions:

The first is \textciteRehm{}, who works on an \nameref{cha:scenarios_enterprise}.
He points out that autonomy is important, but is currently uncertain if a performant approach will be needed later on and whether it will sacrifice some autonomy.
Furthermore, he believes that \ac{UX} is still the most important fact and needs to be preserved even at the expense of autonomy.
On the contrary, the second opinion is from \textciteJovanovic{}, who has worked on a \nameref{cha:scenarios_consumer}.
His concern was that older browsers or smartphones can run into performance issues.
Such issues occur for example, when running Angular and React on the same page.
This is why they ultimately implemented a policy, that only one heavy framework per page is allowed.

The \hyperref[cha:requirement_detail_performance]{Performance} requirement is considered critical from \textciteRehm{} and \textciteSteyer{} and difficult by \textciteJovanovic{}.

