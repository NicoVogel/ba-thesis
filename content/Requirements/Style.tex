%!TEX root = ../../documentation.tex


\section{Style requirement and approaches}\label{cha:requirement_detail_style}

The style requirement category cannot be divided into sub requirements.
Therefore, it is only the \textit{Consistent style (\textbf{Score 15})} requirement and its information.
This information can be grouped into multiple aspects, which will be presented in this section.
First, the general intent, namely style isolation is outlined.
Following that are two specific approaches: utility \ac{CSS} file and component library.
Lastly, practices are presented in this section



\paragraph{Style isolation}

A key part of styling in \ac{MF} applications is isolating the styles between the \acp{MF}.
There are various ways to isolate styles and which approach is used depends on how autonomous the teams should be.
More on team autonomy in the \textit{\nameref{cha:requirement_detail_developer_autonomy}} requirement.

To achieve style isolation, \textcite{Jackson.2019} mentions techniques like \ac{SASS}, \ac{BEM}, \ac{CSS} in \ac{JS}, shadow \ac{DOM} and \ac{CSS} modules can be used.
The last technique was proposed by \textcite{Laug.2018b}.
The first technique, \ac{SASS}, is a preprocessing variance of \ac{CSS}.
It is a superset of \ac{CSS} which enables features like nesting, mixins, and inheritance\footnotemark.
\footnotetext{\url{https://sass-lang.com/guide} (Visited on 08/21/2020)}
Next is \ac{BEM}, that is a naming convention for \ac{CSS} classes\footnotemark.
\footnotetext{\url{http://getbem.com/introduction/} (Visited on 08/21/2020)}
The third is \ac{CSS} in \ac{JS}, which moves styles into \ac{JS}.
Another approach is the shadow \ac{DOM}, a Web Component feature, which utilizes a browser native isolation mechanism.
Finally, \ac{CSS} modules are a step in the build process, that enforce isolation by changing \ac{CSS} class names to be unique\footnotemark.
\footnotetext{\url{https://github.com/css-modules/css-modules} (Visited on 08/21/2020)}

Besides these known approaches, there was a custom approach  mentioned by \textcite{Grijzen.2019}.
He explained that each \ac{MF} has its own namespace which is equivalent to the \ac{MF} name.
Because the project also has the \textit{\nameref{cha:requirement_detail_integration_pagelayout}} requirement, it is possible to apply the namespace to the layout placeholders.
To enable the isolation, the namespace is added as prefix to each \ac{CSS} class within a \ac{MF}.

All these techniques can be used to achieve isolating style, but which technique is used is up to the architect.
For example, \textcite{Dornenburg.2019} believes that \ac{BEM} is only a workaround.



\paragraph{Utility \ac{CSS}}

One approach that was mentioned multiple times, is to use a utility \ac{CSS} in the project.
Such a \ac{CSS} file contains global styles which are shared between all \acp{MF}.
\textciteOlleck{} points out, that such a \ac{CSS} file can either be loaded once by the shell and thus create a tight coupling, or each \ac{MF} loads it by themselves.
The last approach is only possible if the shadow \ac{DOM} is used, because it isolates the styles within the Web Component.
This also allows to load multiple versions of the styles within the same project.
This also keeps coupling low between the \acs{MF}, even though a global \ac{CSS} is used.
\textciteRehm{} mentions, that a utility \ac{CSS} file should contain a spacing system.
A spacing system defines the sizes of different elements, like buttons and paragraphs.
This ensures a consistent look regarding spaces between elements.
Finally, the last comment on this is from \textcite{Laug.2018}.
He suggests avoiding \ac{CSS} rules to be included in the utility \ac{CSS} file which exceed margins and paddings.



\paragraph{Component library}

Another approach, which was mentioned multiple times, is the usage of a component library.
A component library\footnotemark can either be framework specific or agnostic, which uses Web Components.
\footnotetext{\cite[p.~16]{Vesselov.2019}}
\textciteRehm{} points out, that in both cases it should be loaded by the shell to provide it only once for all \acp{MF}.
This leaves the question open, when a component library should be framework specific or agnostic.

There are two scenarios where a framework specific component library is used.
The first was mentioned by \textciteJovanovic{} and it is when Web Components are not available, then the component library needs to be framework specific.
Another reason can be that only one framework is used in the application.
For example, \textciteRehm{} pointed out, that the application he is working on only uses Angular as framework.
Therefore, some shortcuts can be achieved if an Angular component library is used instead of a framework agnostic version.
He further mentions that it could be swapped out with a Web Component library, but there is no need currently.

On the other hand, a framework agnostic approach enables a technology swap later on.
A framework agnostic component library is used by \textciteOlleck{} and \textciteRehm{}.
\citeauthorRehm{} explains, that even though they use an Angular component library for the components that they build, a second technology agnostic component library is provided by the customer.
Therefore, he uses both types at the same time.
However \citeauthorOlleck{} points out, that using a component library based on Web Components can be problematic.
For example, if the version of the Web Component library is upgraded and any \ac{MF} needs the new version, then the entire application needs to be updated.
This is due to the fact, that it is not possible to register a Web Component with the same name twice.
Therefore, a tight coupling is implied between the Web Component library and the \acp{MF}.
\citeauthorOlleck{} points out, that this can be solved by adding the version into the Web Component name.

There is also some general advice for creating a component library.
\textcite{Laug.2018} experienced, that component libraries sometimes may be over engineered.
An example for an over engineered component would be a list component with a select, filter and search feature and the user is supposed to select one item of three.
In this case the filter and search features are not needed, hence over engineered.
He states that in this case these features are not needed and are disturbing.
Therefore, providing basic versions of components can help prevent over engineering.
This is also pointed out by \textcite{Jackson.2019}, where he recommends to focus on basic components.
He points out, that to preserve reusability for complex components, they need to be built without any business logic.



\paragraph{Practices}

The experts mentioned some style practices.
The most named practice is Design System.
A Design System is a collection of components, groupings of components and guidelines.
For further detail, see \cite[p.~16]{Vesselov.2019}.
Design System is used by \textciteRehm{}, \textciteOlleck{}, \textcite{Leitner.2020}, \textcite{Laug.2018}, \textcite{Grijzen.2019}, \textciteSteyer{} and \textciteHuber{}.
In general, \citeauthorSteyer{} noticed an increased interest in the usage of Design Systems.

Besides Design System, there are also the practices like Design Guild and Inner Sourcing.
Design Guild focuses on the idea of collaboratively evolving design.
To achieve this, a regular meeting is conducted, where decisions around design are discussed and decided.
The meeting is attended by one member of each team and this member has the responsibility to share the knowledge in the team.
This principle was successfully utilized by \textcite{Laug.2018}.
He also pointed out, that one result of the Design Guild should be a style guide which provides copy and paste ready code \cite{Laug.2018b}.

On the other hand, Inner Sourcing uses the approach of creating a project internally, which has at least one main contributor and others can contribute via pull requests, therefore mimicking an open source project.
This approach was explained by \textciteSteyer{} and also mentioned by \textcite{Jackson.2019}.



\paragraph{Conclusion}

To achieve consistent style in a \ac{MF} application there is always a tradeoff between autonomy and \ac{UX}.
Various methods to isolate style were outlined, but total isolation is not necessarily what is needed.
Therefore, many experts mentioned a utility \ac{CSS} file.
This file couples \acp{MF} together, but the amount of coupling depends on the styling rules.
A large decision is always whether to use global or local styles, explains \textciteSteyer{}.

Another common approach are component libraries and they are also used in combination with other practices.
\textciteSteyer{} mentions, that a frequent practice nowadays is Design System.
\textcite{Dornenburg.2019} states, that in general styling is about components rather than the entire application.
He continues by pointing out, that the times when one large \ac{CSS} was used to style an application are over and styles apply at the component level.

This requirement is considered crucial by \textciteRehm{} and \textciteSteyer{} and difficult by \textciteOlleck{}.
