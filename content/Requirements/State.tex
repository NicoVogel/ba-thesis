%!TEX root = ../../documentation.tex

\section{State requirements and approaches}\label{cha:requirement_detail_state}

This section explains the state requirements.
The requirements from \textit{\nameref{cha:requirement_detail_state_exchange}} until \textit{\nameref{cha:requirement_detail_state_security}} are described in greater detail, than the requirement \textit{\nameref{cha:requirement_detail_state_internationalization}} due to its score being lower than 5.





\subsection{State exchange}\label{cha:requirement_detail_state_exchange}

The requirement \textit{\nameref{cha:requirement_detail_state_exchange} (\textbf{Score 16})} is about transferring the application context into and between \acp{MF}, so that each \ac{MF} can display content accordingly.
Sharing the context can be achieved with four different approaches.



\paragraph{\ac{URL} context}

One approach is to use the \ac{URL} to share context information.
For instance, a \ac{URL} can contain an \ac{ID} for a user or product, accessible by all \acp{MF}.
\textciteSteyer{} and \textcite{Grijzen.2019} outlined, that this approach allows for deep linking\footnotemark.
\footnotetext{\url{https://www.w3.org/2001/tag/doc/deeplinking.html} (Visited on 09/06/2020)}
Therefore, the application allows the usage of bookmarks or sharing a \ac{URL} with a colleagues.
The approach was also mentioned by \textciteRehm{} and \textcite{Jackson.2019} and is especially useful in case of a \ac{URL} change which results in a \ac{MF} swap.
This swap causes a new \ac{MF} to be loaded, which needs to gain insights about the current context.

However, using the \ac{URL} for state transfer has limitations.
At some point of a \ac{URL} length, the usability drops when sharing a \ac{URL}, which is too long or when saving it as a bookmark.
For example, sharing a \ac{URL} with a length of 10.000 characters is unwieldy.
On the other hand, some context information like security tokens should not be shared via the \ac{URL}.
\textciteRehm{} also mentioned, that if widgets are used, the \ac{URL} is not suited for context transfer.
He explains that the intent of a widget is to be used everywhere and it is not clear what the \ac{URL} looks like for any given \ac{MF}.
Hence, a standard way of constructing the \ac{URL} would be needed.
This would create more coupling and therefore, \citeauthorRehm{} advises against using this approach for widget context transfer.
More information on widgets in the \textit{\nameref{cha:requirement_detail_integration_widget}} requirement.



\paragraph{\ac{HTML} context}
Another approach is to use \ac{HTML} attributes.
This can only be used if the application also uses Web Components.
More about Web Components in the \textit{\nameref{cha:requirement_detail_integration_integration}} requirement.
Web Components support two ways to transfer information.
One is to use \ac{HTML} attributes and the other is via properties.
This approach was mentioned by \textciteRehm{} and \textciteSteyer{}.
More on Web Components in the section \ref{cha:theory_practice_webcomponent}.

The former is achieved, by simply adding values into the \ac{HTML} document into the correct node.
Contrary to that, the latter uses the browser \ac{API} to get the \ac{JS} object of the node and then set the value.
Using \ac{HTML} attributes is simple to use in a framework like, Angular, React or Vue.
These have an extensive templating support and therefore, only little work is needed to implement the state exchange via \ac{HTML} attributes.
However, \ac{HTML} attributes are limited to string values.

\textciteRehm{} suggests, that if there is a need to transfer rich data like an object or array, then it should be transferred via a property.
Properties can also be used to transfer primitive data like strings, numbers and booleans and it is best practice to reflect these as attributes\footnotemark[1]{}.
In case of rich data, however, reflecting the values in attributes is needlessly expensive and thus, should not be done\footnotemark[1]{}.
\footnotetext[1]{\url{https://developers.google.com/web/fundamentals/web-components/best-practices\#attributes-properties} (Visited on 08/19/2020)}

\textciteSteyer{} further suggests defining an interface for the data transfer.
In case of widgets, both the \ac{HTML} attributes and property data transfer offer a valid solution.
This was mentioned by \textciteRehm{}.



\paragraph{Shell provides context}

The third approach to exchange the state is that each \ac{MF} requests the context from the shell.
\textciteOlleck{} suggests providing an \ac{API} which is publicly available from the shell, that has some function to retrieve the current context.
He mentions that it could also be split up, so that there is a function for each context type.
As an example, one function can be used to get language information and another to get user information.
Moreover, he mentions that the number of versions available should be limited to the current and last two versions for instance.

\textciteSteyer{} recommends that the shell should hold some of the context information to pass it on to newly created \acp{MF}.
This recommendation suites both the \ac{HTML} attribute and shell context providing approach. The same applies for information which is passed from the configuration to a \ac{MF}.
This was also mentioned by \textciteOlleck{} in the \textit{\nameref{cha:requirement_detail_integration_configuration}} requirement.



\paragraph{Global Redux store}

The last approach mentioned was the usage of a global Redux store\footnotemark.
\footnotetext{\url{https://redux.js.org/api/store} (Visited on 09/06/2020)}
\textciteJovanovic{} mentions, that a Redux store can contain any data and therefore, it can be used to transfer context information.
He suggests creating a library which is then used for accessing the context information out of the Redux store.
In the project he worked on, they did not create such a library, but documented how to access it and each \ac{MF} was responsible to access it on its own.
Lastly, he encourages to use the Redux store for bug fixing purposes.
A Redux store contains the application flow and therefore reproducing an error is simplified.
Another aspect is to consider the autonomy between \acp{MF}.
To maintain it, either use a prefix for sub stores \cite{Dornenburg.2019} or each \ac{MF} uses its own store \cite{Laug.2018b}.

Regarding the question if a Redux store should be used or not, there are different opinions.
While \textciteJovanovic{}, \textcite{Dornenburg.2019}, \textcite{Laug.2018b} and \textcite{Jackson.2019} suggest using it,  \textciteSteyer{}, \textcite{Leitner.2020} are against it.
\citeauthor{Laug.2018} was against the usage of a Redux store in April of 2018 and supported the approach in November 2018 \cite{Laug.2018b}.
So, there seems to be a diversion of opinions about using a Redux store.



\paragraph{Conclusion}

Sharing context over the \ac{URL} is a simple and effective approach, while using \ac{HTML} attributes is only possible if Web Components are used.
Using the shell to provide the context does introduce a certain amount of coupling between the \acp{MF} and the shell but can be used if Web Components are not an option.
Lastly, the Redux store simplifies the bug fixing process, but seems to be dependent on the architects opinion.
There is no clear indication that the scenario affects the decision on whether to use a Redux store or not.
This requirement is considered critical by \textciteRehm{}, \textciteSteyer{}, \textciteJovanovic{} and difficult by \textciteOlleck{}.





\subsection{Intercommunication}\label{cha:requirement_detail_state_intercommunication}

The \textit{\nameref{cha:requirement_detail_state_intercommunication} (\textbf{Score 10})} requirement evolves around the need of information exchange between \acp{MF}, which is beyond state exchange.
There are two approaches to achieve this requirement.
Either the \acp{MF} communicate in the client with each other or indirectly over the server.



\paragraph{Client communication}

In case that the communication should be handled in the client itself, a message bus approach is commonly used.
Using such an approach has a low coupling between the \acp{MF}.
There are two implementations for this approach.
Either using Custom Events which are browser native or a library is included which simplifies or enhances the process.
More on that in the section \ref{cha:theory_practice_customevents}.

The Custom Event implementation was mentioned by \textciteSteyer{}, \textcite{Jackson.2019}, \textcite{Grijzen.2019} and \textcite{Dornenburg.2019}.
Sending Events to manage lifecycle methods is also a possibility explained in the \textit{\nameref{cha:requirement_detail_integration_lifecycle}} requirement.
Its main advantage is that no additional library is needed, but in some scenarios, this is not sophisticated enough.

Therefore, \citeauthorSteyer{} and \textcite{Leitner.2020} suggested using a library like RxJS to handle intercommunication.
\citeauthorSteyer{} mentioned as an example a Behavior Subject.
It requires an initial value and whenever a \ac{MF} subscribes, the latest value is directly emitted, as well as whenever some \ac{MF} publishes a new value\footnotemark{}.
\footnotetext{\url{https://www.learnrxjs.io/learn-rxjs/subjects/behaviorsubject} (Visited on 08/19/2020)}



\paragraph{Server communication}

Another approach to manage intercommunication is by using the backend as a communication platform.
\citeauthorSteyer{} and \citeauthor{Dornenburg.2019} mention that a WebSocket connection of server send events could be used for this purpose.
\citeauthorSteyer{} continues by pointing out that this is rarely used but is necessary for the \textit{Hyperlink Integration}.
A variation of this approach was suggested by not only \citeauthorSteyer{} and \citeauthor{Dornenburg.2019}, but also \textciteJovanovic{} and \textcite{Laug.2018}.
If the backend architecture \ac{BFF} is used, then these dedicated microservices can be used to communicate with each other.
\ac{BFF} is further discussed in the Theory chapter in \textit{\nameref{cha:theory_extensions_bff}}.
Moreover, \citeauthor{Laug.2018} points out that this approach allows a practice called \ac{CDC} testing.
He also draws attention to the fact, that there is no good \ac{CDC} Testing approach for the client side.
That is the reason, he was using the server communication approach in the first place.
\citeauthorJovanovic{} also mentions that in the end, he did not use this approach, because it was too much overhead.



\paragraph{Conclusion}

In case of intercommunication, there is either the option for a hardly testable approach in the client-side or a testable approach with overhead on the server-side.
The server-side approach is arguably a workaround for the fact, that there is currently no good client-side \ac{CDC} testing, which was pointed out by \textcite{Laug.2018}
Apart from that, this requirement is considered critically by \textciteOlleck{}.





\subsection{Security}\label{cha:requirement_detail_state_security}

The next requirement is \textit{\nameref{cha:requirement_detail_state_security} (\textbf{Score 5})}, which addresses authentication and authorization for an application.
All experts mentioned a token-based authentication technique and therefore, only approaches for this technique are explained.
The first thing to consider is, where to handle authentication and authorization.
\textcite{Laug.2018} believes that authentication is a critical task and that it should be handled by the shell, while authorization should be handled by each \ac{MF}.
Hence, sharing the authentication in the shell.
More about sharing logic over the shell in the \textit{\nameref{cha:requirement_detail_integration_sharedlogic}} requirement.

\textcite{Laug.2018} pointed out, that in his scenario it is not a huge problem if authorization goes wrong, because it is an internal \nameref{cha:scenarios_enterprise}.
He explains, that by providing the application over the same domain, \ac{HTTP} cookies can be used.
This is only possible because everything is provided via the same origin.
Besides \citeauthor{Laug.2018}, this was also mentioned by \textciteSteyer{} and he further points out, that it is the current recommended way of handling tokens\footnotemark.
\footnotetext{\url{https://developer.mozilla.org/en-US/docs/Web/HTTP/Cookies} (Visited on 08/20/2020)}

There are two other approaches.
The first is pointed out by \textciteSteyer{} and is sharing the token via a message bus.
In this case, the shell informs each \ac{MF} about the token.
More about message bus approaches in the \textit{\nameref{cha:requirement_detail_state_exchange}} requirement.
On the other hand, the second approach is using the local storage, which was the recommended way prior to \ac{HTTP} cookies.
This was mentioned by \textciteJovanovic{}.

The experts only mentioned token sharing approaches, apart form \textcite{Laug.2018}.
He pointed out what the responsibilities in the application are.
This requirement is considered critical by \textciteSteyer{}.





\subsection{Other requirements}

In the state requirement category, only the \textit{\nameref{cha:requirement_detail_state_internationalization}} requirement has a score of less than 5.

\paragraph{Internationalization}\label{cha:requirement_detail_state_internationalization}

The \textit{\nameref{cha:requirement_detail_state_internationalization} (\textbf{Score 2})} requirement addresses either a language change and \ac{MF} changes based on geographical location.
\textciteSteyer{} points out, that in case of a language change feature, the approach he used was always the same in any project:
Some part in the application must raise an event that informs all \acp{MF} about the language change and each \ac{MF} is responsible to change its language whenever requested.
This event is transmitted via a message bus approach.
More about message bus approaches in the \textit{\nameref{cha:requirement_detail_state_exchange}} requirement.

The second feature is about \ac{MF} changes based on the geographical location.
This was explained by \textciteMezzalira{}.
He points out, that the application can swap out \acp{MF} based on the country it is accessed from.
The application he is working on can only support this feature, because it uses a configuration file.
More about configuring the application in the \textit{\nameref{cha:requirement_detail_integration_configuration}} requirement.
