%!TEX root = ../../documentation.tex


\section{Developer requirements and approaches}\label{cha:requirement_detail_developer}

In the following, the requirements \textit{\nameref{cha:requirement_detail_developer_autonomy}}, \textit{\nameref{cha:requirement_detail_developer_experience}} and \textit{\nameref{cha:requirement_detail_developer_tools}} are described in greater detail as well as other requirements, however in less detail due to their score.

Also, it is important to note, that the \textit{\nameref{cha:requirement_detail_integration_configuration}} requirement is described in the section \ref{cha:requirement_detail_integration} instead of here, because it is part in two categories.
It is the only requirement that fits into two categories and is already fully explained in the Integration chapter.





\subsection{Autonomy}\label{cha:requirement_detail_developer_autonomy}

The \textit{\nameref{cha:requirement_detail_developer_autonomy} (\textbf{Score 13})} requirement focuses on the independence aspect of each \ac{MF}.
This relates directly to the developer freedom.
Also, the \textit{\nameref{cha:requirement_detail_developer_autonomy}} requirement can be split into the different facades coupling, communication, release, and technology agnostic.



\paragraph{Coupling}

One of the first things which is relevant for \textit{\nameref{cha:requirement_detail_developer_autonomy}}, is coupling.
Coupling is a term used to indicate if software parts are dependent on each other.
In case of \ac{MFA}, coupling is not desired between the individual \acp{MF}.
This would hinder the individual deployment benefit.
However, sometimes there is coupling required.
For example, in the configuration file for the shell application.
More about configuration in the \textit{\nameref{cha:requirement_detail_integration_configuration}} requirement.
There is some coupling because there are \ac{MF} specific entries.
But the coupling is rather low, because it is only in the configuration file and it probably is a one-time setup.
This approach of keeping coupling low and use a one-time setup if possible, was mentioned by \textcite{Laug.2018} and \textcite{Dornenburg.2019}.
\citeauthor{Laug.2018} also points out, that sharing code between \acp{MF} should be avoided.
He believes that it is better to provide \textit{copy-paste} ready code for each team, rather then to share code.
Another part which creates coupling is sharing styles.
This is addressed in the \textit{\hyperref[cha:requirement_detail_style]{Consistent style} } requirement, but generally the opinions of the experts diverge.
For example, \citeauthor{Laug.2018} also believes that styles should be \textit{copy-paste} ready components instead of sharing actual code, while \textciteOlleck{} used a global utility \ac{CSS} file with multiple versions.



\paragraph{Communication}

Another part of autonomy is reflected in the communication.
There are two kinds of communication for \ac{MFA}.
The first is the intercommunication, which is discussed in more detail in the \textit{\nameref{cha:requirement_detail_state_intercommunication}} requirement.
All experts that mention intercommunication approaches, used a message bus solution.
It has low coupling and is very flexible.
The second communication type is the communication between the \ac{MF} and backend.
The most named approach for this communication is \ac{BFF}, which is explained in the section \ref{cha:theory_extensions_bff}.
It was mentioned by \textciteSteyer{}, \textcite{Jackson.2019}, \textcite{Leitner.2020} and \textcite{Laug.2018}.
\citeauthor{Leitner.2020} believes that using \ac{BFF} is mandatory.
He further explains, that using \ac{BFF} also implies that one \ac{MF} and its \ac{BFF} microservice are considered as one unit.
\citeauthorSteyer{} also recommends to always use the \ac{BFF} pattern if possible.



\paragraph{Release}

\textit{\nameref{cha:requirement_detail_developer_autonomy}} is not only about developing, but also about deployment and release processes.
The experts mentioned that \ac{CI/CD} plays a large role for deploying and releasing \acp{MF}.
\textciteRehm{} suggests, that each \ac{MF} should have its own \ac{CI/CD} pipeline.
Another aspect which was mentioned, is the independent release cycle.
This is one of the core benefits of \ac{MFA} and \citeauthorRehm{} mentions, that a release should be quick.
Therefore, both \textciteMezzalira{} and \citeauthorRehm{} believe, that a \ac{CI/CD} pipeline must be quick for a good developer experience.
\citeauthorMezzalira{} states, that a \ac{CI/CD} should run within minutes.
Additionally, \citeauthorRehm{} proposes that each \ac{MF} should be released in its own versioned Docker container\footnotemark, which enables practices like Blue-Green deployment.
\footnotetext{\url{https://www.docker.com/resources/what-container} (Visited on 09/06/2020)}
Another aspect which is important to allow such practices is the configuration file of the application.
More about configuration in the \textit{\nameref{cha:requirement_detail_integration_configuration}} requirement.
Finally, \textcite{Laug.2018} supports \citeauthorRehm{} statement and adds, that if \ac{BFF} is used, then both the \ac{MF} and \ac{BFF} microservice could be in one Docker container.



\paragraph{Technology agnostic}

Before concluding this requirement, it is important to discuss the aspect of supporting technology agnostic \acp{MF}.
A \ac{MF} application can either use only one frontend framework or multiple.
Depending on the scenario restrictions, a framework agnostic approach can be realized or not.
For example, \textciteJovanovic{} was working on a project where initially React and Angular was used, but it resulted in performance issues.
They limited the number of heavy frameworks used per page to one.
More about the performance impact of frameworks in the \textit{\hyperref[cha:requirement_detail_performance]{Performance}} requirement.

\textciteJovanovic{} suggests, that a default technology stack should be defined and it should be used by every team.
But he points out, that using this approach is not necessarily the best solution.
He believes, that if there is a better technology available, it should be used instead.
A similar statement was made by \textcite{Laug.2018b}.
He mentions that normally developers stay in their comfort zone and thus use the same technology stack as the other teams.
Therefore, the fear to end up supporting dozens of frameworks is mostly unjustified.
Lastly, \textciteHuber{} also supports the idea of an open stack which allows for a potential technology shift later on.

Another approach was used by \textcite{Grijzen.2019}.
In the project he worked for, they only support one technology stack.
The advantage is, that they can define one standard to align all projects.
This is further improved by a \ac{CLI} they created to allow for bootstrapping new \acp{MF}.
He also mentions that it makes it easier for developers to change teams.
Finally, another facade of technology agnostic approaches regarding styling is addressed  in the \textit{\hyperref[cha:requirement_detail_style]{Consistent style}} requirement.



\paragraph{Conclusion}

\textit{\nameref{cha:requirement_detail_developer_autonomy}} is a base principle of \ac{MFA}.
It should be preserved, by keeping the coupling between the \acp{MF} low.
There are multiple approaches that can be used to achieve that.
For example, instead of sharing code it should be \textit{copy-paste} read.
Another is choosing an appropriate intercommunication method.
These address the actual coding process, but there are also concepts or process related aspects which preserve autonomy.
Namely, the concept \ac{BFF} and individual \ac{CI/CD} pipelines per \ac{MF}.
To conclude the \textit{\nameref{cha:requirement_detail_developer_autonomy}} requirement, there is also the question if the application should be technology agnostic or not.
Most of the experts used a technology agnostic approach, so that if needed, another technology can be used.
On the other hand, using one technology stack has the advantage of better utilizing tools.





\subsection{Experience}\label{cha:requirement_detail_developer_experience}

The \textit{\nameref{cha:requirement_detail_developer_experience} (\textbf{Score 8})} requirement addresses the fact, that \ac{MFA} adds a lot of complexity to an application.
Therefore, there should be some measurements to simplify the developers work to maintain a good developer experience.

One step to ensure a good developer experience is a quick developer feedback, which is ensured with a fast \ac{CI/CD} pipeline.
This was mentioned by \textciteMezzalira{} and \textcite{Grijzen.2019}.
\citeauthorMezzalira{} points out, that the time from implementing a new feature until the developer receives his feedback needs to be minimal.
He also mentions that it needs constant work to maintain fast feedback.
This is also addressed in the \textit{\nameref{cha:requirement_detail_developer_autonomy}} requirement in the release paragraph.
\citeauthor{Grijzen.2019} adds to \citeauthorMezzalira{}, that everything should be automated, to remove unnecessary work from the developer.

Another step mentioned by \textcite{Grijzen.2019} was to provide a library for common actions.
The shell could provide this library and it handles tasks like navigation and persist or receive \ac{URL} state.
Further possibilities which address styling are a component library and utility \ac{CSS}.
This was mentioned by \textciteRehm{} and \textciteHuber{}.

Lastly, \textciteSteyer{} proposes that the \ac{MF} teams should not work on the shell or the integration.
Instead, there should be a separate team responsible for this.
On the other hand, \textcite{Laug.2018} believes, that the shell should not change over time and thus making \citeauthorSteyer{}s statement obsolete.
Finally, \textciteMezzalira{} thinks, that this is a difficult requirement.




\subsection{Tools}\label{cha:requirement_detail_developer_tools}

The \textit{\nameref{cha:requirement_detail_developer_tools} (\textbf{Score 5})} requirement is about tools that simplify the development of a \ac{MF} application.
Even though this is considered critical by \textciteJovanovic{}, there is rather low input overall about tooling.
The most extensive tool was presented by \textcite{Grijzen.2019}.
He showed a \ac{CLI} tool, that they created for their application.
It handles tasks like project setup, build setup and deployment pipelines, and it ensures that everything works out of the box.
The only other tool mentioned are Zeplin and Storybook, which were proposed by \textciteRehm{}.
Both tools fall into the \ac{UI} development sector.
Zeplin shows which components are used within a page, as well as other helpful information\footnotemark.
\footnotetext{\url{https://zeplin.io/why-zeplin} (Visited on 08/22/2020)}
On the other hand, Storybook is a tool to create \ac{UI} components in isolation and document them\footnotemark.
\footnotetext{\url{https://storybook.js.org/docs/react/get-started/introduction} (Visited on 08/23/2020)}





\subsection{Other requirements}

The following requirements have a score less than 5.



\paragraph{Debugging}\label{cha:requirement_detail_developer_debugging}

The \textit{\nameref{cha:requirement_detail_developer_debugging} (\textbf{Score 3})} requirement addresses approaches which simplifies debugging in a \ac{MF} application.
A specialty of a \ac{MF} application, is that it consists of multiple individual \acp{MF}, which are composed together by the shell to one application.
This raises the question on how to debug a distributed system.
The experts provided some advice on how to tackle this problem.

\textciteSteyer{} and \textcite{Laug.2018} propose, that there should be a standalone mode for each \ac{MF}.
The standalone version should allow for development, debugging, testing and integration of the \ac{MF}.
When running a \ac{MF} in standalone mode, the development should feel and work no different than ordinary applications.
\citeauthor{Laug.2018} points out, that he used a different shell (mock shell) to integrate the \ac{MF} in the same way as it would be in the actual shell.
Each \ac{MF} has its own mock shell and is part of the \ac{MF}.
Therefore, this allows that it can be modified as needed to fit any of the testing and developing needs.
This approach is also used by \textciteRehm{}.

This only covers the frontend, but sometimes the backend is also needed for development.
\textciteRehm{} mentioned an approach he used to keep this need simple for the developers.
To provide a backend for local development of a \ac{MF}, they used Kubernetes port forwarding\footnotemark.
\footnotetext{\url{https://kubernetes.io/docs/tasks/access-application-cluster/port-forward-access-application-cluster/} (Visited on 09/06/2020)}
This forwarded an instance of the backend service from a development server to a local port.
For each \ac{MF} they provided a command which sets up the connection, therefore eliminating the need to start the backend services on the local machine.





\paragraph{Testing}\label{cha:requirement_detail_developer_testing}

The \textit{\nameref{cha:requirement_detail_developer_testing} (\textbf{Score 3})} requirement is about testing a \ac{MF}.
Testing a \ac{MF} should be no different from a regular application, except for the \ac{MF} specific parts, like integration and intercommunication.

In case of integration testing, there were two approaches mentioned.
The first is from \textcite{Dornenburg.2019}.
He explained that the configuration that is used by the shell to load each \ac{MF}, can be changed to load one \ac{MF} locally and the rest from a development server.
Therefore, only the shell and one \ac{MF} need to be started for the integration test.
The second approach was mentioned by \textcite{Laug.2018b}.
He proposes a shell integration method, which checks if the actual exports of the \ac{MF} matches with its configuration.
It only checks if it is available and not if it works as intended.
He points out, that it is about how to get certainty in the integration process before a user gets to see the new release.
More on integration in the \textit{\nameref{cha:requirement_detail_integration_integration}} requirement.


The other testing aspect is intercommunication and it is about \ac{CDC} testing (see \ref{cha:theory_practice_cdctesting}).
\textcite{Jackson.2019} believes, that \ac{CDC} testing should be used to validate communication between \acp{MF}, but he does not mention how it should be done.
On the other hand \textcite{Laug.2018} points out, that there is no good client \ac{CDC} testing approach for \ac{MF} applications.
Therefore, he did not use intercommunication in the client, but rather used the backend to handle this need.
More on that in the \textit{\nameref{cha:requirement_detail_state_intercommunication}} requirement.
