%!TEX root = ../documentation.tex

% \chapter{Architecture}\label{cha:architecture}
% % titel vorschlag: architektur empfehlung
% % man könnte theoretisch auch 

% \textcite[p.~13]{Fielding.2000} said that, properties are induced by the set of constraints within an architecture.
% % see: https://youtu.be/WlsetrZgSMQ?t=1123
% Which means that architectural design choices for a project will yield certain characteristics for the solution.
% In chapter \ref{cha:requirement} possible requirements are discussed and these are the constraints for the architecture.
% Thus, the shell architecture needs to be able to support these constraints.
% It is important to mention, that there are also custom business constraints which can effect the shell.
% This results in the need for a flexible architecture.

% There two possible architectural techniques to achieve the needed flexibility for a shell application.
% The first one is a base implementation which is modular and allows for plugins and the second is creating a framework which solve the common problems.




