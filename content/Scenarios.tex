%!TEX root = ../documentation.tex

\chapter{Application scenarios}\label{cha:scenarios}

One aspect while researching the \ac{MFA} became apparent, namely that it is relevant to consider that requirements change depending on the application scenario.
There are several requirements for \ac{MF} applications, and it is not apparent which requirements are needed or useful for any given application.
Therefore, the intent of the second question in the expert  interviews (see \ref{cha:appendix_expertinterview_questions}) is to collect information on different \ac{MF} application scenarios and connect the information with requirements.
Initially, the intent of the second question was to gain intel about real world examples and later link the examples to requirements.
The goal was to provide in-depth information of which requirement is needed for which kind of application.
Therefore, the experts should name suitable scenarios for the \ac{MFA}.
But all experts believe that it can not be pinpointed to scenarios, but rather criteria.
As a result, the experts' information for the first question are summarized in \ref{cha:scenarios_types} and the named criteria are summarized in \ref{cha:scenarios_criteria}.





\section{General application types for micro frontend architecture}\label{cha:scenarios_types}

Based on the expert interviews and conference presentations, three general application types have been identified where \ac{MFA} can be used.
These general types are collections of statements.



\paragraph{Enterprise Application}\label{cha:scenarios_enterprise}

The first application type is an \nameref{cha:scenarios_enterprise}.
Generally this application type is used company-internal and the devices for access are typically known or at least the minimum performance is known.
Therefore, they can be device specific and do not necessarily support multiple platforms.
This narrows the supported device scope and hence more platform specific approaches can be used.
Furthermore, such an application is likely to be used frequently or it is at least accessed via a fast and stable internet connection, like for example a \ac{LAN} connection.
As a result an extensive caching approach is recommendable and the application size is not as relevant as in the other application types.
Another aspect is that multiple companies could work on the same application, which requires enforcing autonomy of the teams.
In the end, such an application can have a user base ranging from a few experts to an enterprise-wide tool with 15.000 or more users.
This type of application was mostly named by experts and in conference presentations, namely by \textcite{Laug.2018}, \textciteRehm{}, \textciteHuber{}, \textciteSteyer{} and \textciteOlleck{}.



\paragraph{Consumer Application}\label{cha:scenarios_consumer}

The second application type is the \nameref{cha:scenarios_consumer}.
Other than the \nameref{cha:scenarios_enterprise}, this application type is used from any kind of network or device.
Moreover, it is to be expected that these devices or networks are slow.
As a result, performance aspects like application size and request amount to the backend are important to ensure a good \ac{UX}.
In the expert interview with \textciteJovanovic{}, he pointed out that performance is the most critical requirement for these applications.
Also, the application probably has a varying active user base, which means that caching is only effective for the active user base.
Depending on the application, it could even be used from a smart TVs for instance.
\textciteMezzalira{} pointed out, that the browsers in such devices support less features than a regular desktop or mobile browser.
For example Web Components are currently not supported.



\paragraph{Off-the-shelf Application}\label{cha:scenarios_offshelf}

An \nameref{cha:scenarios_offshelf} is a product from a company which is distributed and customized for customers.
It is similar to an \nameref{cha:scenarios_enterprise}, but due to the customization requirement it has a slightly shifted focus.
For example, the application needs to be extendable via custom build features from the customer or even third-party companies.
To enable such behavior, the application probably needs to provide an interface for these features.
Another fact which changes from the \nameref{cha:scenarios_enterprise} is that the performance of the user devices is probably not known.
Therefore, it is important to invest into performance to ensure a good \ac{UX} for the customers \cite{Grijzen.2019}.
As a result, this application type can be placed between \nameref{cha:scenarios_enterprise} and \nameref{cha:scenarios_consumer}.





\section{Typical criteria to consider micro frontend architectures}\label{cha:scenarios_criteria}

After describing the general application types, the experts also pointed out which criteria needs to apply to at least consider a \ac{MFA}.
The criteria mostly mentioned is a large team size.
It was named by \textciteRehm{}, \textciteOlleck{}, \textciteJovanovic{} and \textciteMezzalira{}.
\citeauthorMezzalira{} provides more detail by indicating that at least 50 developers should work on the application.

Another criterion is bound to the project or team structure.
If a project is developed in an agile environment, follows the DevOps culture, or has cross-functional teams in general, then using the \ac{MFA} could maintain the development speed.
This was pointed out by \textciteRehm{}.
\textciteJovanovic{} suggested that the \ac{MFA} should be used if it suits the organization.
This includes the named structures, but possibly also other practices like \textit{Extreme Programming}.
The next criteria named by \textciteOlleck{} is, that the project consists of strongly separated business functionalities.

Consequently, he points out that enforcing the strong separation with the \ac{MFA} will most likely maintain development speed over time.
Besides that, \textciteOlleck{} and \textciteJovanovic{} mention that a big application is also a criteria.
Moreover, \citeauthorOlleck{} mentioned the criteria, that if multiple companies work on the same project, the \ac{MFA} will reduce the communication needed between these companies making an independent deployment simpler.
Lastly, an \nameref{cha:scenarios_enterprise} is well suited for the \ac{MFA}.
\textciteSteyer{} points out, that the micro-architecture trend also started with \nameref{cha:scenarios_enterprise}s, which started to implement the microservice architecture first.

Another criterion or rather scenario is gradually converting from a monolith to a \ac{MF} application.
\textciteSteyer{} mentioned that it is not a common scenario, while \textciteJovanovic{} worked on such a project and \textcite{Jackson.2019} also supports the idea.
Hence, this scenario can be considered as a criterion for considering the \ac{MFA}.
Finally, \citeauthorSteyer{} experienced that nearly all \ac{MF} applications were green field projects, which he supervised.
A green field project is a project which starts from scratch, therefore the term \enquote{green field}, because there is nothing implemented yet.

To sum up this chapter, there are three application types for \ac{MFA}, namely \textit{\nameref{cha:scenarios_enterprise}}, \textit{\nameref{cha:scenarios_consumer}} and \textit{\nameref{cha:scenarios_offshelf}}.
Still, they practically cover all applications.
Therefore, some criteria should be met in order to start considering using the architecture.

\begin{itemize}
    \item Large teams
    \item Suiting the project or organization structure
    \item Strongly separated business functionalities
    \item Application size
    \item Multiple companies working on the same application
\end{itemize}
