%!TEX root = ../documentation.tex
\chapter{Introduction}\label{cha:introdction}

% introduction (background)
At the end of 2016, the company ThoughtWorks did mention the architecture \textit{micro frontend} for the first time \cite{ThoughtWorks.2016}.
This architecture is the counterpart of the server-side architecture \textit{microservices} and intends to bring the same benefits from the server-side to the client-side.
It is not a new architecture, but it gained traction in 2019 \cite{Betts.2020}.
This leaves the question open on why it is gaining traction now.
The reason for choosing this architecture differs between projects.
Experts agree that project characteristics, such as an extensive application or having many developers work on it, are indicators for at least considering the micro frontend architecture \cite{Vogel.2020.Olleck} - \cite{Vogel.2020.Rehm}.

% Generally, there are some effects that slow down the development process, like big teams or having an extensive application.
% Such problems are addressed in the backend with the \textit{microservice} architecture and now also in the frontend with the \textit{micro frontend} architecture.

The \textit{micro frontend} architecture is generally about splitting a frontend into multiple independent frontends and assemble them back to one application at some point.
There are two possible approaches to integrate the frontends for web based applications.
The first approach is called \textit{Web Approach} (or \textit{Hyperlink Integration}), which uses hyperlinks to connect the frontends.
On the other hand, the second approach is called the \textit{Container Approach} (or \textit{Shell}), which has some form of the host application that integrates the frontends.
The \textit{Container Approach} has the three integration variances \textit{Build-Time Integration}, \textit{Server-Side Composition}, and \textit{Client-Side Composition} \cite[p.~69ff.]{Wenzel.2020} \cite{Leitner.2020}. %min 22:50


% what is the problem
Since the \textit{micro frontend} architecture is new in the software community and many companies have not used it until now, there is still a high number of details hidden.
Moreover, for each partial problem, there are multiple approaches, and each has its pros and cons.
All the knowledge required to start working with micro frontends provides an entry barrier for companies which are new to this architecture.


% goal
This thesis revolves around the requirements and approaches of the integration technique \textit{Client-Side Composition}.
The goal is to evaluate if there is a need for a generalized container approach that uses this integration approach.
% Therefore, other integration techniques are not part of this work.
In order to achieve this goal, it is important to first outline the characteristics of an application that utilizes this type of integration.
This is done via a requirement analysis, which reveals the general needs.
After general needs of such an application are known an evaluation can take place to check whether a generalized implementation is actually required and what it needs to fulfill.



% motivation
Even though the term \textit{micro frontends} first appeared in 2016, there is already an implementation available called \textit{Single-SPA}, which was first released 2015.
% https://github.com/single-spa/single-spa/tags?after=v1.2.0
This implementation aims to provide the features required for the \textit{Client-Side Composition} approach.
An expert states, that \textit{Single-SPA} has an opinionated architecture \cite{Vogel.2020.Mezzalira}.

This work evaluates on the basis of a requirement analysis which requirements must be fulfilled by a shell application for different scenarios.
Based on the results, this thesis aims to provide a fresh view of the topic and then evaluates whether one should rely on \textit{Single-SPA}, or alternatively develop an own customized solution.



% - method
Currently, the literature regarding \textit{micro frontends} is still limited.
Therefore, expert interviews are conducted, which will be the base of this work.
The interviewees are either experts with the \textit{micro frontend} architecture (which are \textciteOlleck{}, \textciteMezzalira{}, \textciteSteyer{} and \textciteJovanovic{}) or are knowledgeable in the field of frontend development, with current efforts to learn and implement the \textit{micro frontend} architecture (\textciteRehm{} and \textciteHuber{}).
This separation into two different types of experts is useful to get a broader view on the topic.
\citeauthorJovanovic{} points out, that the \textit{micro frontend} architecture is rather new overall and is perceived as complex by many.
Therefore, it is important to invest in lowering the entry barrier of this architecture.

For data collection, the interviews are guided by questions that represent different topics.
The questions are asked in such a way that the experts can also present their view of the topics.
It is also mandatory to build up a knowledge base of the topics, before interviewing the experts \cite[p.~31]{AlexanderBognerBeateLittigWolfgangMenz.2009}.
Apart from that, to ensure the comparability of the expert statements, it is important to get more insights about their organizational context \cite[p.~35]{AlexanderBognerBeateLittigWolfgangMenz.2009}.
Therefore, the first two questions in the interview are used to gain insights of their background (see \ref{cha:appendix_expertinterview_questions}).
The results are collected and combined into scenarios that are explained in chapter \ref{cha:scenarios}.
The other questions guide the interview through the topics and the final question is used to get the expert's opinion on the main question of this thesis.



% - structure of thesis
This thesis starts with the fundamentals of \textit{micro frontends} (chapter \ref{cha:Theory}).
Next, some suitable scenarios for the \textit{micro frontend} architecture are described(chapter \ref{cha:scenarios}), which are addressed at various points through out the work.
The following main part of this thesis is the requirement analysis (chapter \ref{cha:requirement}).
This leads to the evaluation (chapter \ref{cha:evaluation}) and eventually the conclusion (chapter \ref{cha:conclusion}).
