%!TEX root = ../documentation.tex

\pagestyle{empty}

\renewcommand{\abstractname}{\langabstract} % Text für Überschrift

\begin{abstract}
 
	Micro frontend is a concept to compose one frontend from multiple decoupled frontend applications.
	It is typically used in web applications.
	There are multiple ways to implement the micro frontend concept.
	One approach is client-side-composition, which is implemented with JavaScript.
	For this approach, the composition in the browser is handled by the so-called \enquote{shell}.
	Shells have varying requirements depending on the resulting application size and complexity.
	This thesis evaluates if there is a need for a generalized shell and how its architecture could look like.
	First, requirements for a shell were identified and their approaches are outlined.
	This was achieved by conducting expert interviews and the resulting requirements are ordered depending on their importance.
	The interviews also provided intel on possible scenarios for a micro frontend application.
	Based on the requirements and scenarios an evaluation was conducted if a generalized shell application is needed.
	This was mainly affirmed by the experts and hence, a possible concept for each scenario is outlined.
	These are then compared with each other in terms of a metric which was extracted from the requirements.
	The extracted metric shows the affiliation of the shell and micro frontend application to general attitude directions.
	Finally, it is evaluated if the known micro frontend framework \textit{Single-SPA} can be used to implement the proposed concepts.

\end{abstract}
