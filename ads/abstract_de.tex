%!TEX root = ../documentation.tex

\pagestyle{empty}

\renewcommand{\abstractname}{\langabstract} % Text für Überschrift

\begin{abstract}

	Micro-Frontend ist ein Konzept zur Zusammenstellung eines Frontends aus mehreren entkoppelten Frontendapplikationen.
	Es wird typischerweise in Web-Anwendungen verwendet.
	Es gibt mehrere Möglichkeiten, das Konzept der Mikro-Frontends zu implementieren.
	Ein Ansatz ist die clientseitige Komposition, die mit JavaScript implementiert wird.
	Bei diesem Ansatz wird die Komposition im Browser von der so genannten "Shell" übernommen.
	Shells haben je nach der daraus resultierenden Anwendungsgröße und Komplexität unterschiedliche Anforderungen.
	In dieser Arbeit wird untersucht, ob ein Bedarf für eine generalisierte Shell besteht und wie deren Architektur aussehen könnte.
	Zunächst wurden die Anforderungen an eine Shell identifiziert und ihre Ansätze skizziert.
	Dies wurde durch die Durchführung von Experteninterviews erreicht, und die daraus resultierenden Anforderungen je nach ihrer Bedeutung geordnet.
	Die Interviews lieferten auch Informationen über mögliche Szenarien für eine Micro-Frontend-Anwendung.
	Auf der Grundlage dieser Anforderungen und Szenarien wurde evaluiert, ob eine generalisierte Shell-Anwendung erforderlich ist.
	Dies wurde von den Experten im Wesentlichen bejaht, so dass für jedes Szenario ein mögliches Konzept skizziert wird.
  Anhand einer Metrik werden diese Konzepte, die aus den Anforderungen extrahiert wurden miteinander verglichen.
	Die extrahierte Metrik zeigt die Zugehörigkeit der Shell- und Micro-Frontend-Anwendung zu allgemeinen Einstellungsrichtungen.
	Schließlich wird evaluiert, ob das bekannte Micro-Frontend-Framework Single-SPA zur Implementierung der vorgeschlagenen Konzepte verwendet werden kann.
\end{abstract}
