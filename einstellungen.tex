%%%%%%%%%%%%%%%%%%%%%%%%%%%%%%%%%%%%%%%%%%%%%%%%%%%%%%%%%%%%%%%%%%%%%%%%%%%%%%%
% Einstellungen
%
% Hier können alle relevanten Einstellungen für diese Arbeit gesetzt werden.
% Dazu gehören Angaben u.a. über den Autor sowie Formatierungen.
%
%
%%%%%%%%%%%%%%%%%%%%%%%%%%%%%%%%%%%%%%%%%%%%%%%%%%%%%%%%%%%%%%%%%%%%%%%%%%%%%%%


%%%%%%%%%%%%%%%%%%%%%%%%%%%%%%%%%%%% Sprache %%%%%%%%%%%%%%%%%%%%%%%%%%%%%%%%%%%
%% Aktuell sind Deutsch und Englisch unterstützt.
%% Es werden nicht nur alle vom Dokument erzeugten Texte in
%% der entsprechenden Sprache angezeigt, sondern auch weitere
%% Aspekte angepasst, wie z.B. die Anführungszeichen und
%% Datumsformate.
\setzesprache{en} % oder en
%%%%%%%%%%%%%%%%%%%%%%%%%%%%%%%%%%%%%%%%%%%%%%%%%%%%%%%%%%%%%%%%%%%%%%%%%%%%%%%%

%%%%%%%%%%%%%%%%%%%%%%%%%%%%%%%%%%% Angaben %%%%%%%%%%%%%%%%%%%%%%%%%%%%%%%%%%%
%% Die meisten der folgenden Daten werden auf dem
%% Deckblatt angezeigt, einige auch im weiteren Verlauf
%% des Dokuments.

% student 1
\setzemartrikelnr{5704145}
\setzekurs{STG-TINF17-D}
\setzeautor{Nico Vogel}

\setzetitel{Evaluation of micro frontend shells for a prototypical generalized shell implementation}
%\setzetitel{Logfileanalyse mit Apache{\textsuperscript{TM}} Hadoop\textsuperscript{{\textregistered}} MapReduce}
\setzedatumAbgabe{September 2020}
\setzefirma{Capgemini}
\setzefirmenort{Stuttgart}
\setzeabgabeort{Stuttgart}
\setzeabschluss{Bachelor of Science}
\setzestudiengang{Informatik}
\setzedhbw{Stuttgart}
\setzebetreuer{Pirmin Rehm}
\setzegutachter{Martin Spörl}
\setzezeitraum{06/15/2020 - 09/07/2020}
\setzearbeit{Bachelor thesis}
%%%%%%%%%%%%%%%%%%%%%%%%%%%%%%%%%%%%%%%%%%%%%%%%%%%%%%%%%%%%%%%%%%%%%%%%%%%%%%%%

%%%%%%%%%%%%%%%%%%%%%%%%%%%% Literaturverzeichnis %%%%%%%%%%%%%%%%%%%%%%%%%%%%%%
%% Bei Fehlern während der Verarbeitung bitte in ads/header.tex bei der
%% Einbindung des Pakets biblatex (ungefähr ab Zeile 110,
%% einmal für jede Sprache), biber in bibtex ändern.
\newcommand{\ladeliteratur}{%
    \addbibresource{bibliography.bib}
    %\addbibresource{weitereDatei.bib}
}
%% Zitierstil
%% siehe: http://ctan.mirrorcatalogs.com/macros/latex/contrib/biblatex/doc/biblatex.pdf (3.3.1 Citation Styles)
%% mögliche Werte z.B numeric-comp, alphabetic, authoryear
\setzezitierstil{numeric-comp}
%%%%%%%%%%%%%%%%%%%%%%%%%%%%%%%%%%%%%%%%%%%%%%%%%%%%%%%%%%%%%%%%%%%%%%%%%%%%%%%%

%%%%%%%%%%%%%%%%%%%%%%%%%%%%%%%%% Layout %%%%%%%%%%%%%%%%%%%%%%%%%%%%%%%%%%%%%%%
%% Verschiedene Schriftarten
% laut nag Warnung: palatino obsolete, use mathpazo, helvet (option scaled=.95), courier instead
\setzeschriftart{lmodern} % palatino oder goudysans, lmodern, libertine

%% Paket um Textteile drehen zu können
%\usepackage{rotating}
%% Paket um Seite im Querformat anzuzeigen
%\usepackage{lscape}

%% Seitenränder
\setzeseitenrand{2.5cm}

%% Abstand vor Kapitelüberschriften zum oberen Seitenrand
\setzekapitelabstand{20pt}

%% Spaltenabstand
\setzespaltenabstand{10pt}
%%Zeilenabstand innerhalb einer Tabelle
\setzezeilenabstand{1.5}
%%%%%%%%%%%%%%%%%%%%%%%%%%%%%%%%%%%%%%%%%%%%%%%%%%%%%%%%%%%%%%%%%%%%%%%%%%%%%%%%

%%%%%%%%%%%%%%%%%%%%%%%%%%%%% Verschiedenes %%%%%%%%%%%%%%%%%%%%%%%%%%%%%%%%%%%
%% Farben (Angabe in HTML-Notation mit großen Buchstaben)
\newcommand{\ladefarben}{%
    \definecolor{LinkColor}{HTML}{00007A}
    \definecolor{ListingBackground}{HTML}{FCFAFB}
}
%% Mathematikpakete benutzen (Pakete aktivieren)
\usepackage{amsmath}
\usepackage{amssymb}

%% Programmiersprachen Highlighting (Listings)
\newcommand{\listingsettings}{%
    \lstset{%
        language=Java,			% Standardsprache des Quellcodes
        numbers=left,			% Zeilennummern links
        stepnumber=1,			% Jede Zeile nummerieren.
        numbersep=5pt,			% 5pt Abstand zum Quellcode
        numberstyle=\tiny,		% Zeichengrösse 'tiny' für die Nummern.
        breaklines=true,		% Zeilen umbrechen wenn notwendig.
        breakautoindent=true,	% Nach dem Zeilenumbruch Zeile einrücken.
        postbreak=\space,		% Bei Leerzeichen umbrechen.
        tabsize=2,				% Tabulatorgrösse 2
        basicstyle=\ttfamily\footnotesize, % Nichtproportionale Schrift, klein für den Quellcode
        showspaces=false,		% Leerzeichen nicht anzeigen.
        showstringspaces=false,	% Leerzeichen auch in Strings ('') nicht anzeigen.
        extendedchars=true,		% Alle Zeichen vom Latin1 Zeichensatz anzeigen.
        captionpos=b,			% sets the caption-position to bottom
        backgroundcolor=\color{ListingBackground}, % Hintergrundfarbe des Quellcodes setzen.
        xleftmargin=0pt,		% Rand links
        xrightmargin=0pt,		% Rand rechts
        frame=single,			% Rahmen an
        frameround=ffff,
        rulecolor=\color{darkgray},	% Rahmenfarbe
        fillcolor=\color{ListingBackground},
        keywordstyle=\color[rgb]{0.133,0.133,0.6}\bfseries,
        commentstyle=\color{Sepia},
        stringstyle=\color{red}
    }
}
%%%%%%%%%%%%%%%%%%%%%%%%%%%%%%%%%%%%%%%%%%%%%%%%%%%%%%%%%%%%%%%%%%%%%%%%%%%%%%%%

%%%%%%%%%%%%%%%%%%%%%%%%%%%%%%%% Eigenes %%%%%%%%%%%%%%%%%%%%%%%%%%%%%%%%%%%%%%%
%% Hier können Ergänzungen zur Präambel vorgenommen werden (eigene Pakete, Einstellungen)

% xcolor muss mit optionen vor pdfpages geladen werden
\usepackage[usenames,dvipsnames,table,xcdraw]{xcolor} 	%xcolor für HTML-Notation

\usepackage{pdfpages}

\usepackage{xargs}
\usepackage[colorinlistoftodos,prependcaption,textsize=tiny]{todonotes}
\newcommandx{\unsure}[2][1=]{\todo[linecolor=red,backgroundcolor=red!25,bordercolor=red,#1]{#2}}
\newcommandx{\change}[2][1=]{\todo[linecolor=blue,backgroundcolor=blue!25,bordercolor=blue,#1]{#2}}
\newcommandx{\info}[2][1=]{\todo[linecolor=OliveGreen,backgroundcolor=OliveGreen!25,bordercolor=OliveGreen,#1]{#2}}
\newcommandx{\improvement}[2][1=]{\todo[linecolor=Plum,backgroundcolor=Plum!25,bordercolor=Plum,#1]{#2}}


% Table formatting
\newcommand\setrow[1]{\gdef\rowmac{#1}#1\ignorespaces}
\usepackage{longtable}
\usepackage{makecell}
% \usepackage{booktabs}
% \usepackage{tablefootnote}


% interview settings
\usepackage[T1]{fontenc}
\usepackage{xparse}
\usepackage{enumitem}

% transcript configuration
\setlist[description]{
    font={\sffamily\bfseries},
    labelsep=0pt,
    labelwidth=\transcriptlen,
    leftmargin=\transcriptlen,
}
\newlength{\transcriptlen}
\NewDocumentCommand {\setspeaker} { mo } {%
    \IfNoValueTF{#2}
    {\expandafter\newcommand\csname#1\endcsname{\item[#1:]}}%
    {\expandafter\newcommand\csname#1\endcsname{\item[#2:]}}%
    \IfNoValueTF{#2}
    {\settowidth{\transcriptlen}{#1}}%
    {\settowidth{\transcriptlen}{#2}}%
}
\setspeaker{NicoVogel}[V]
% How much of a gap between speakers and text?
\addtolength{\transcriptlen}{1em}%


% create commands for the interviewer names
\NewDocumentCommand{\defineInterview}{m}{
    \expandafter\newcommand\csname citeauthor#1\endcsname{#1}
    \expandafter\newcommand\csname textcite#1\endcsname{#1 \cite{Vogel.2020.#1}}
}
% don't forget to add an empty backet behind each command
\defineInterview{Olleck}
\defineInterview{Jovanovic}
\defineInterview{Mezzalira}
\defineInterview{Steyer}
\defineInterview{Huber}
\defineInterview{Rehm}


% indent
\usepackage{scrextend}
\NewDocumentCommand{\requirementText}{m}{
    \begin{addmargin}[1em]{2em}% 1em left, 2em right
        #1
    \end{addmargin}
}

% URL
\usepackage[hyphenbreaks]{breakurl}
\usepackage[hyphens]{url}

\let\oldfootnote\footnote
\renewcommand{\footnote}{\unskip\oldfootnote}% Remove any skips inserted before \footnote

% allow to reference the name and number of something
\newcommand*{\fullref}[1]{\hyperref[{#1}]{\ref*{#1} \nameref*{#1}}}

% check and x marks
\usepackage{pifont}% http://ctan.org/pkg/pifont
\newcommand{\cmark}{\ding{51}}%
\newcommand{\xmark}{\ding{55}}%
